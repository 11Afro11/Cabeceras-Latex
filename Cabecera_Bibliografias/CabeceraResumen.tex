\documentclass[10pt,spanish, landscape, twocolumn]{article}

\usepackage[spanish]{babel}
\usepackage[utf8]{inputenc}
\usepackage{amsmath, amsthm}
\usepackage{amsfonts, amssymb, latexsym}
\usepackage{enumerate}
\usepackage[usenames, dvipsnames]{color}
\usepackage{colortbl}
\usepackage[landscape]{geometry} % puedes modificar los márgenes con este paquete
\usepackage{minted}

\usepackage[bookmarks=true,
            bookmarksnumbered=false, % true means bookmarks in
                                     % left window are numbered
            bookmarksopen=false,     % true means only level 1
                                     % are displayed.
            colorlinks=true,
            linkcolor=webblue]{hyperref}
\definecolor{webgreen}{rgb}{0, 0.5, 0} % less intense green
\definecolor{webblue}{rgb}{0, 0, 0.5}  % less intense blue
\definecolor{webred}{rgb}{0.5, 0, 0}   % less intense red

\setlength{\parindent}{0pt}
\setlength{\parskip}{1ex plus 0.5ex minus 0.2ex}

%Definimos autor y título
\title{Los comandos \LaTeX\ que más uso y de los que nunca me acuerdo} 
\author{Marta Gómez}

\begin{document}
\maketitle % si queréis más espacio en la primera página eliminad esta línea.

\section{Matemáticos}

\begin{minted}[frame=single, label={overbrace y underbrace}]{latex}
\begin{displaymath}
    \underbrace{\overbrace{a+b+c}^6 \cdot 
    \overbrace{d+e+f}^9}
    _\text{meaning of life} = 42
\end{displaymath}
\end{minted}

\begin{displaymath}
    \underbrace{\overbrace{a+b+c}^6 \cdot \overbrace{d+e+f}^9}_\text{meaning of life} = 42
\end{displaymath}

\begin{minted}[frame=single, label={binom}]{latex}
    \begin{displaymath}
        \binom{n}{k} = \binom{n-1}{k} + 
                       \binom{n-1}{k-1}
    \end{displaymath}
\end{minted}

\begin{displaymath}
    \binom{n}{k} = \binom{n-1}{k} + \binom{n-1}{k-1}
\end{displaymath}
\\

\begin{minted}[frame=single, label={math\_fonts}]{latex}
    \begin{displaymath}
        \mathcal{R} \qquad \mathfrak{R} \qquad 
        \mathbb{R}
    \end{displaymath}
\end{minted}

\begin{displaymath}
    \mathcal{R} \qquad \mathfrak{R} \qquad \mathbb{R}
\end{displaymath}

\begin{minted}[frame=single, label={bigs}]{latex}
    \begin{displaymath}
        \big( \Big( \bigg( \Bigg( \qquad\
        \big{ \Big{ \bigg{ \Bigg{ \qquad\
        \big\Downarrow \Big\Downarrow 
            \bigg\Downarrow \Bigg\Downarrow
    \end{displaymath}
\end{minted}

\begin{displaymath}
    \big( \Big( \bigg( \Bigg( \qquad\
    \big\{ \Big\{ \bigg\{ \Bigg\{ \qquad\
    \big\Downarrow \Big\Downarrow 
        \bigg\Downarrow \Bigg\Downarrow
\end{displaymath}

\end{document}